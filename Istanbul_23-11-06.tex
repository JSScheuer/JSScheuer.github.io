\documentclass{beamer}
\usetheme{Boadilla}      % or try Darmstadt, Madrid, Warsaw, ...
  \usecolortheme{beaver} % or try albatross, beaver, crane, ...
  \usefonttheme{default}  % or try serif, structurebold, ...
  \setbeamertemplate{navigation symbols}{}
  \setbeamertemplate{caption}[numbered]
\usepackage{color,soul}
\definecolor{red}{rgb}{1,0,0}
\setulcolor{red}
\usepackage{wrapfig}
\usepackage{wasysym}
\usepackage{tikz}

\usepackage[T1]{fontenc}
\usepackage[polish]{babel}
\usepackage[utf8]{inputenc}


\usepackage{amssymb}
\usepackage{amsmath}
\usepackage{amsthm}
\usepackage{esint}
\usepackage{enumerate}
\usepackage{multimedia}

\usepackage{autonum}



%Symbols
\newcommand{\ti}{\tilde}
\newcommand{\wt}{\widetilde}
\newcommand{\wh}{\widehat}
\newcommand{\bs}{\backslash}
\newcommand{\cn}{\colon}
\newcommand{\sub}{\subset}
\newcommand{\ov}{\overline}
\newcommand{\mr}{\mathring}


\newcommand{\bbN}{\mathbb{N}}
\newcommand{\bbZ}{\mathbb{Z}}
\newcommand{\bbQ}{\mathbb{Q}}
\newcommand{\bbR}{\mathbb{R}}
\newcommand{\bbC}{\mathbb{C}}
\newcommand{\bbS}{\mathbb{S}}
\newcommand{\bbH}{\mathbb{H}}
\newcommand{\bbK}{\mathbb{K}}
\newcommand{\bbD}{\mathbb{D}}
\newcommand{\bbB}{\mathbb{B}}
\newcommand{\bbE}{\mathbb{E}}
\newcommand{\bbM}{\mathbb{M}}


\newcommand{\8}{\infty}

%Greek letters
\newcommand{\al}{\alpha}
\newcommand{\be}{\beta}
\newcommand{\ga}{\gamma}
\newcommand{\de}{\delta}
\newcommand{\ep}{\epsilon}
\newcommand{\ka}{\kappa}
\newcommand{\la}{\lambda}
\newcommand{\om}{\omega}
\newcommand{\si}{\sigma}
\newcommand{\Si}{\Sigma}
\newcommand{\ph}{\phi}
\newcommand{\vp}{\varphi}
\newcommand{\ze}{\zeta}
\newcommand{\vt}{\vartheta}
\newcommand{\Om}{\Omega}
\newcommand{\De}{\Delta}
\newcommand{\Ga}{\Gamma}
\newcommand{\Th}{\Theta}
\newcommand{\La}{\Lambda}
\newcommand{\Ph}{\Phi}
\newcommand{\Ps}{\Psi}

%Mathcal Letters
\newcommand{\cL}{\mathcal{L}}
\newcommand{\cT}{\mathcal{T}}
\newcommand{\cA}{\mathcal{A}}
\newcommand{\cW}{\mathcal{W}}
\newcommand{\cH}{\mathcal{H}}
\newcommand{\cS}{\mathcal{S}}
\newcommand{\cD}{\mathcal{D}}
\newcommand{\cB}{\mathcal{B}}
\newcommand{\cF}{\mathcal{F}}
\newcommand{\cN}{\mathcal{N}}
\newcommand{\cU}{\mathcal{U}}
\newcommand{\cV}{\mathcal{V}}
\newcommand{\cO}{\mathcal{O}}
\newcommand{\cI}{\mathcal{I}}
\newcommand{\cK}{\mathcal{K}}
\newcommand{\cR}{\mathcal{R}}
\newcommand{\cP}{\mathcal{P}}






%Mathematical operators
\newcommand{\INT}{\int_{\O}}
\newcommand{\DINT}{\int_{\d\O}}
\newcommand{\Int}{\int_{-\infty}^{\infty}}
\newcommand{\del}{\partial}
\newcommand{\n}{\nabla}
\newcommand{\II}[2]{\mrm{II}\br{#1,#2}}
\newcommand{\fa}{\forall}
\newcommand{\rt}{\sqrt}




\newcommand{\ip}[2]{\left\langle #1,#2 \right\rangle}
\newcommand{\fr}[2]{\frac{#1}{#2}}
\newcommand{\tfr}[2]{\tfrac{#1}{#2}}
\newcommand{\x}{\times}

\DeclareMathOperator{\dive}{div}
\DeclareMathOperator{\id}{id}
\DeclareMathOperator{\pr}{pr}
\DeclareMathOperator{\Diff}{Diff}
\DeclareMathOperator{\supp}{supp}
\DeclareMathOperator{\graph}{graph}
\DeclareMathOperator{\osc}{osc}
\DeclareMathOperator{\const}{const}
\DeclareMathOperator{\dist}{dist}
\DeclareMathOperator{\loc}{loc}
\DeclareMathOperator{\tr}{tr}
\DeclareMathOperator{\Rm}{Rm}
\DeclareMathOperator{\Rc}{Rc}
\DeclareMathOperator{\Sc}{R}
\DeclareMathOperator{\grad}{grad}
\DeclareMathOperator{\ad}{ad}
\DeclareMathOperator{\argmax}{argmax}
\DeclareMathOperator{\vol}{vol}
\DeclareMathOperator{\Area}{Area}
\DeclareMathOperator{\sgn}{sgn}
\DeclareMathOperator{\Rad}{Rad}
\DeclareMathOperator{\nul}{null}
\DeclareMathOperator{\ind}{ind}
\DeclareMathOperator{\diam}{diam}









%Environments


%\newcommand{\Theo}[3]{\begin{#1}\label{#2} #3 \end{#1}}
\newcommand{\pf}[1]{\begin{proof}#1 \end{proof}}
\newcommand{\eq}[1]{\begin{equation}\begin{alignedat}{2} #1 \end{alignedat}\end{equation}}
\newcommand{\IntEq}[4]{#1&#2#3	 &\quad &\text{in}~#4,}
\newcommand{\BEq}[4]{#1&#2#3	 &\quad &\text{on}~#4}
\newcommand{\br}[1]{\left(#1\right)}
\newcommand{\abs}[1]{\lvert #1\rvert}
\newcommand{\enum}[1]{\begin{enumerate}[(i)] #1 \end{enumerate}}
\newcommand{\enu}[1]{\begin{enumerate}[(a)] #1 \end{enumerate}}
\newcommand{\Matrix}[1]{\begin{pmatrix} #1 \end{pmatrix}}



%Logical symbols
\newcommand{\Ra}{\Rightarrow}
\newcommand{\ra}{\rightarrow}
\newcommand{\hra}{\hookrightarrow}
\newcommand{\Iff}{\Leftrightarrow}
\newcommand{\mt}{\mapsto}

%Fonts
\newcommand{\mc}{\mathcal}
\newcommand{\tit}{\textit}
\newcommand{\mrm}{\mathrm}

%Spacing
\newcommand{\hp}{\hphantom}
\newcommand{\q}{\quad}




%\beamerdefaultoverlayspecification{<+->}

 
\title[Stability for quermassintegrals]{Stability for the quermassintegral inequalities in the hyperbolic space}
\author[J. Scheuer]{Julian Scheuer\\ (Goethe University Frankfurt)
}

\date[Istanbul, 06/11/23]{{\bf{International workshop on geometry of submanifolds}}}
 
\begin{document}

\maketitle


%%%%%%%%%%%%%%%%%%%%%%%%%%%%%%%%%%%%

\begin{frame} \setbeamercovered{invisible}
\frametitle{Closed Soap Bubbles}

\begin{itemize}
		\item[] {\textbf{Isoperimetric problem}}: Determine properties of area minimising surface, given volume constraint. 
		\item[] Round spheres in $\bbR^{n+1}$ are unique closed minimisers of
		\eq{\cR(\Om) = \frac{\Area^{\fr{n+1}{n}}(\del\Om)}{\mrm{Volume}(\Om)}}
 	\item[] Standard variational methods:
	\begin{itemize}
	\item Minimisers of $\cR$ have {\bf{constant mean curvature}} 
	\eq{H=\tr(A) = \sum_{i=1}^{n}\ka_{i}}
	$(\ka_{i})$ are eigenvalues of the Weingarten map $A$, principal curvatures.
	\end{itemize}
\end{itemize}	
\end{frame}

\begin{frame} \setbeamercovered{invisible}
\frametitle{Alexandrov's theorem}

\begin{center}
{\textbf{Is a closed embedded constant mean curvature (CMC) hypersurface of $\bbR^{n+1}$ necessarily a sphere? }}
\end{center}


\begin{itemize}
\item[] Answer: {\textbf{YES!} (Alexandrov\footnote{\emph{A characteristic property of spheres}, Ann. Mat.
  Pura Appl. \textbf{58} (1962), no.~4, 303--315.})}
	\begin{itemize}
		\item Proof: Reflection across moving planes and the maximum principle.
		\item We are going to see another elegant proof today. 
	\end{itemize}
\item[] Relaxed CMC condition: Suppose for some $\de>0$, on a hypersurface $M$
\eq{n-\de\leq H\leq n+\de.}
\item[] Can we conclude
\eq{\dist(M,S)\leq C\ep}
for the unit sphere $S$, a constant $C$ and where 
\eq{\lim_{\de\ra 0}\ep(\de)= 0~?}
\end{itemize}
\end{frame}

\begin{frame} \setbeamercovered{invisible}
\frametitle{The question of stability}

\begin{theorem}[Giulio Ciraolo and Luigi Vezzoni\footnotetext{\emph{A sharp quantitative version of
  {A}lexandrov's theorem via the method of moving planes}, J. Eur. Math. Soc.
  \textbf{20} (2018), no.~2, 261--299.}]
Let $\Om$ be a smooth domain with connected boundary, then {\bf{$\del\Om$ lies within an annulus of thickness $C\osc(H)$}}. $C$ depends on $\abs{\del\Om}$ and a lower bound for interior and exterior balls.
\end{theorem}


 Generalization to spaceforms and other curvature functions
\eq{F=F(\ka_{i})}
was given by Ciraolo/Roncoroni/Vezzoni.\footnote{\emph{Quantitative
  stability for hypersurfaces with almost constant curvature in space forms},
  Ann. Mat. Pura Appl. \textbf{200} (2021), no.~5, 2043--2083.}


\end{frame}




\begin{frame} \setbeamercovered{invisible}
\frametitle{The question of stability}
%\begin{center}
%{\textbf{Is a closed embedded almost constant mean curvature hypersurface of $\bbR^{n+1}$ necessarily close to a sphere? }}
%\end{center}

\begin{theorem}[Rolando Magnanini and Giorgio Poggesi\footnotetext{\emph{On the stability for
  {A}lexandrov's soap bubble theorem}, J. Anal. Math. \textbf{139} (2019),
  no.~1, 179--205.}]
Let $\Om$ be a smooth domain with connected boundary, then $\del\Om$ lies within an annulus of thickness at most $C\|H-H_{0}\|_{L^{1}(\del\Om)}^{\tau_{n}}$, where
\eq{H_{0} = \fr{n}{n+1}\fr{\abs{\del\Om}}{\abs{\Om}},}
 $\tau_{n}$ is a dimensional constant and $C$ depends on few geometric quantities, such as interior and exterior ball conditions.

\end{theorem}
\end{frame}


\begin{frame} \setbeamercovered{invisible}
\frametitle{Integral approach}

\begin{itemize}
\item[] Key ingredient (A. Ros): Use {\bf{Reilly's integral identity}}.
\item[] For $C^{2}$-functions $f$ on a domain $\Om\sub\bbR^{n+1}$, with $f_{|\del\Om}=\const$:
\eq{\int_{\Om}(\De f)^{2}-\int_{\Om}\abs{\n^{2}f}^{2}=\int_{\del\Om}H(\del_{\nu}f)^{2}.}
\item[] {\bf{Cauchy-Schwarz-deficit}}:
\eq{\abs{\mr{\n}^{2}f}^{2}=\abs{\n^{2}f}^{2}-\tfr{1}{n+1}(\De f)^{2},}
Then
\eq{\int_{\Om}\abs{\mr{\n}^{2}f}^{2}=\fr{n}{n+1}\int_{\Om}(\De f)^{2}-\int_{\del\Om}H(\del_{\nu}f)^{2}.}
Solve 
\eq{\De f &= 1\q \mbox{in}~\Om\\
		f &= 0 \q \mbox{on}~\del\Om.}
\end{itemize}
\end{frame}

\begin{frame} \setbeamercovered{invisible}
\frametitle{Integral approach}
\begin{itemize}
\item[] Then 
\eq{\int_{\Om}\abs{\mr{\n}^{2}f}^{2}&=\fr{n}{(n+1)\vol(\Om)}\br{\int_{\Om}\De f}^{2}-\int_{\del\Om}H(\del_{\nu}f)^{2}\\
					&=\fr{n}{(n+1)\vol(\Om)}\br{\int_{\del\Om}\del_{\nu}f}^{2}-\int_{\del\Om}H(\del_{\nu}f)^{2}\\
					&\leq \fr{n}{(n+1)}\fr{\Area(\del\Om)}{{\vol(\Om)}}\int_{\del\Om}(\del_{\nu}f)^{2}-\int_{\del\Om}H(\del_{\nu}f)^{2}\\
					&\equiv\int_{\del\Om}(H_{0}-H)(\del_{\nu}f)^{2}.}


\item[] The function $f-q$ is harmonic, where
\eq{q = \fr{1}{2(n+1)}\abs{x-c}^{2}-R.}
\item[] Magnanini/Poggesi: Good estimates for $f-q$.
\end{itemize}
\end{frame}


\begin{frame} \setbeamercovered{invisible}
\frametitle{New general shortcut}

\begin{theorem}[Level set stability\footnotetext{\emph{Stability from rigidity via umbilicity, (2021), {\href{https://arxiv.org/abs/2103.07178}{arxiv:2103.07178}}.}}]
Let $n\geq 2$, $M\sub \bbR^{n+1}$ closed hypersurface, $\abs{M}=1$. Let $\cU$ be one-sided neighbourhood of $M$, foliated by level sets of $f\in C^{2}(\bar\cU)$, 
\eq{\bar \cU = \bigcup_{0\leq t\leq \max\abs{f}}M_{t},\q M_{t} = \{\abs{f}=t\},} 
 with $f_{|M}=0$ and $\abs{\n f}_{|\bar\cU}>0$.  Let $p>n$ and $\max_{0\leq t\leq \max\abs{f}}\|A\|_{p,M_{t}}\leq C_{0}.$
%Then there exists a constant $C = C(n,p,C_{0})$,
Then
%\eq{\br{\int_{\cU}\abs{\mr{\bar\n}^{2} f}^{p}}^{\fr{1}{p+1}}<\fr{1}{C}\min(\max\abs{f},\min\abs{\bar\n f})^{\fr{p}{p+1}}}
%implies
\eq{\dist(M,\cS)\leq \fr{C(n,p,C_{0})}{\min(\max\abs{f},\min\abs{\n f})^{\fr{p}{p+1}}} \br{\int_{\cU}\abs{\mr{\n}^{2} f}^{p}}^{\fr{1}{p+1}}  } 
for a sphere $\cS$, provided the RHS is small.
\end{theorem} 


\end{frame}




\begin{frame} \setbeamercovered{invisible}
\frametitle{Some Remarks}
\begin{itemize}
\item[] This also works in {\bf{conformally flat Riemannian manifolds}}.
\item[] The exponent $1/(p+1)$ is worse than the constant $\tau_{n}$ in Magnanini/Poggesi's CMC stability problem.
\item[] But, the result is universal among many geometric problems.
\item[] Note in particular, that {\bf{$f$ is not assumed to solve any PDE.}}
\vspace{1cm}



\end{itemize}
\end{frame}


\begin{frame} \setbeamercovered{invisible}
\frametitle{Some words about the proof: Almost umbilicity I}

\begin{itemize}
%\item Key: The curvature of every regular level set $M$ of $f$ can be expressed in terms of $\n^{2}f$:
\item[] If $h$ is the $2^{nd}$ fundamental form of the boundary, then.
\eq{\n^{2}f_{| M}=-\abs{\n f}h.}
\item[] Hence
\eq{\abs{\n f}^{2}\abs{\mr{A}}^{2}=\abs{\mr\n^{2}f_{|TM}}^{2}-\tfr{1}{n}(\mr{\n}^{2}f(\nu,\nu))^{2},  }
\item[] where $\mr{A}$ is the tracefree part of the second fundamental form, 
\eq{\abs{\mr{A}}^{2}=c_{n}\sum_{i<j}(\ka_{i}-\ka_{j})^{2}.} 
\end{itemize}
\end{frame}


\begin{frame} \setbeamercovered{invisible}
\frametitle{Some words about the proof: Almost umbilicity I}
\begin{itemize}
\item[] Hence $\mr{A}$ can be controlled by $\mr\n^{2} f$.
\item[] Classical result from hypersurface theory, due to Darboux:
\eq{\mr{A}=0\q\Ra\q M = \mbox{Sphere}.}
\item[] Stability versions available and due to De Lellis/M\"uller, Topping, Grosjean and recently...
\end{itemize}
\end{frame}

\begin{frame} \setbeamercovered{invisible}
\frametitle{Some words about the proof: Almost umbilicity I}

\begin{theorem}[Antonio De Rosa, Stefano Gioffr\'e\footnotetext{\emph{Absence of bubbling phenomena for
  non-convex anisotropic nearly umbilical and quasi-{E}instein hypersurfaces},
  J. Reine Angew. Math. \textbf{780} (2021), 1--40.}]
Let $M\sub \bbR^{n+1}$ be a closed hypersurface, $p>n$ and 
\eq{|M| = |\bbS^{n}|,\q \|A\|_{L^{p}(M)}\leq c_{0}.}
Then there exist $C = C(n,p,c_{0})>0$, such that: if
\eq{\|\mr{A}\|_{L^{p}(M)}\leq \tfr 1C,}
then there exists $c\in \bbR^{n+1}$, such that $M - c$ is a graph over the sphere, 
\eq{\psi\cn \bbS^{n} \ra M,\q \psi(x) = e^{f(x)}x+c,}
 \eq{\|f\|_{W^{2,p}(\bbS^{n})}\leq C\|\mr{A}\|_{L^{p}(M)}.}
\end{theorem}

 %The Co-area formula relates (locally around $M$) $\|\mr A\|_{L^{2}(M)}$ with $\|\mr{\n }^{2}f\|_{L^{2}(\cU)}$.
\end{frame}




\begin{frame} \setbeamercovered{invisible}
\frametitle{Almost umbilicity II: Quermassintegral inequalities }
\begin{itemize}
	\item[] Classical Steiner formula for convex $\Om\sub\bbR^{n+1}$:
\eq{\abs{\bar\Om+\ep B}=\sum_{k=0}^{n+1}\binom{n+1}{k}W_{k}(\Om)\ep^{k}\q\fa \ep\geq 0.}
\item[] Here
\eq{W_{0}(\Om)=\abs{\Om},\q W_{k}(\Om)=\fr{1}{n+1}\int_{\del\Om}H_{k-1}\q 1\leq k\leq n;}
$H_{k}$ are the normalized elementary symmetric polynomials and $H_{0}=1$.
\item[]
The classical Alexandrov-Fenchel inequalities say that
\eq{\fr{W_{k+1}(\Om)}{W_{k+1}(B)}\geq \br{\fr{W_{k}(\Om)}{W_{k}(B)}}^{\fr{n-1-k}{n-k}},\q 0\leq k\leq n-1,}
with equality precisely if $\Om$ is a ball. \end{itemize}
\end{frame}

\begin{frame} \setbeamercovered{invisible}
\frametitle{Almost umbilicity II: Quermassintegral inequalities}
\begin{itemize}
	\item[] P. Guan/J. Li\footnote{\emph{The quermassintegral inequalities for
  k-convex starshaped domains}, Adv. Math. \textbf{221} (2009), no.~5,
  1725--1732.}: Generalization of the {\bf{QM-inequalities to starshaped and $k$-convex hypersurfaces}}, i.e.
	\eq{\mbox{principal curvatures}\in \Ga_{k} = \{H_{1}>0,\dots, H_{k}>0\}.}
	\item[] For this result we get stability:
	
	
\end{itemize}
\end{frame}

\begin{frame} \setbeamercovered{invisible}
\frametitle{Almost umbilicity II: Quermassintegral inequalities}

\begin{theorem}[Stability in non-convex QM-inequalities\footnotetext{\emph{Stability from rigidity via umbilicity, (2021), {\href{https://arxiv.org/abs/2103.07178}{arxiv:2103.07178}}.}}]
Let $n\geq 2$ and $1\leq k\leq n$. Let $M^{n}\sub\bbM_{0}$ be a closed and starshaped $C^{2}$-hypersurface with interior $\rho$-ball condition such that $\ka\in \Ga_{k}$. Then there exists a constant $C$, depending on $n$, $\abs{M}$ and on {\bf upper bounds for $\rho^{-1}$} and $(\min_{M}\dist(\ka,\del\Ga_{k}))^{-1}$, such that
\eq{\fr{W_{k+1}(\Om)}{W_{k+1}(B)}-\br{\fr{W_{k}(\Om)}{W_{k}(B)}}^{\fr{n-k}{n-k+1}}<C^{-1}}
 implies, for a sphere $S$,
\eq{\dist(M,S)\leq C\br{\fr{W_{k+1}(\Om)}{W_{k+1}(B)}-\br{\fr{W_{k}(\Om)}{W_{k}(B)}}^{\fr{n-k}{n-k+1}}}^{\fr{1}{2(n+1)}}. }
\end{theorem}

\end{frame}

\begin{frame} \setbeamercovered{invisible}
\frametitle{Almost umbilicity II: Quermassintegral inequalities}
\begin{itemize}
	\item[] We use the {\bf{curvature flow}}
		\eq{\label{GLF}\del_{t}x=\br{\fr{H_{k-1}}{H_{k}}-u}\nu.}
	\item[] Convergence to a round sphere settled by Claus Gerhardt\footnote{\emph{Flow of nonconvex hypersurfaces into spheres}, J. Differ.
  Geom. \textbf{32} (1990), no.~1, 299--314.}.
	\item[] This flow preserves $\ti W_{k}(\Om)$ (normalized $W_{k}$) and decreases $\ti W_{k+1}(\Om)$.
	\item[] We estimate a particular space-time integral:
	\eq{\label{pf:AF-1}\int_{0}^{\8}\int_{M_{t}}\br{\fr{H_{k+1}H_{k-1}}{H_{k}}-H_{k}}dt&=\int_{0}^{\8}\int_{M_{t}}\br{\fr{H_{k+1}H_{k-1}}{H_{k}}-uH_{k+1}}dt\\
					&=c_{n,k}\int_{0}^{\8}\fr{d}{dt}\ti W_{k+1}(\Om_{t})dt\\
					&=1-\ti W_{k+1}(\Om)\\
					&=\ti{W}_{k}(\Om)^{\fr{n-k-1}{n-k}}-\ti W_{k+1}(\Om).
}
\end{itemize}
\end{frame}
%
%\begin{frame} \setbeamercovered{invisible}
%\frametitle{Further applications - AF III}
%\begin{itemize}
%	\item We use the {\it{curvature flow}}
%		\eq{\label{GLF}\del_{t}x=\br{\fr{H_{k}}{H_{k+1}}-u}\nu,}
%		where $x\cn [0,\8)\x \bbS^{n-1}\ra \bbR^{n}$.
%	\item Long time existence and smooth convergence to a round sphere settled by C. Gerhardt and J. Urbas in the 90's.
%	\item This flow preserves $\ti W_{k+1}(\Om)$ (normalized $W_{k}$) and increases $\ti W_{k}(\Om)$.
%	\item We estimate a particular space-time integral:
%	\eq{\label{pf:AF-1}\int_{0}^{\8}\int_{M_{t}}\br{\fr{H_{k}^{2}}{H_{k+1}}-H_{k-1}}~dt&=\int_{0}^{\8}\int_{M_{t}}\br{\fr{H_{k}^{2}}{H_{k+1}}-uH_{k}}~dt\\
%					&=c_{n,k}\int_{0}^{\8}\fr{d}{dt}\ti W_{k}(\Om_{t})~dt\\
%					&=1-\ti W_{k}(\Om)\\
%					&=\ti{W}_{k+1}(\Om)^{\fr{n-k}{n-1-k}}-\ti W_{k}(\Om).
%}
%\end{itemize}
%\end{frame}


\begin{frame} \setbeamercovered{invisible}
\frametitle{Almost umbilicity II: Quermassintegral inequalities}
\begin{itemize}
	\item[] Hence the right hand side controls the {\it{Newton-MacLaurin-deficit}}:
	\item[] Recall that
	\eq{H_{k+1}H_{k-1}\leq H_{k}^{2}}
	with equality precisely at umbilic points.
	\item[]
Apply the associated quantitative estimates:
\eq{\fr{\abs{\mr{A}}^{2}}{c(n,1/H_{k})}\leq H_{k}^{2}-H_{k+1}H_{k-1}\q\fa 1\leq k\leq n-2.}
	\item[] Further elementary arguments give $L^{2}$-control on $\abs{\mr{A}}^{2}$.
	\item[] Almost-umbilicity finishes the proof.
\end{itemize}
\end{frame}



\begin{frame} \setbeamercovered{invisible}
\frametitle{Almost umbilicity III: Quermassintegrals in hyperbolic space }
\begin{itemize}
	\item[] For a bounded domain $\Omega$ in hyperbolic space $\mathbb{H}^{n+1}$ with $C^{2}$ boundary $M= \partial \Omega$, the $k^{th}$ quermassintegrals $W_{k}$ is defined inductively:
\eq{
W_{k+1}(\Omega)= \frac{1}{n+1}\int_{M}E_{k}(\kappa)d\mu - \frac{k}{n+2-k}W_{k-1}(\Omega)}
where
\eq{W_{0}(\Omega) = \abs{\Om},\q
W_{1}(\Omega)= \frac{1}{n+1}\lvert M\rvert.
}
	\item[] The extra term comes from ambient hyperbolic curvature, which effects the evolution of the outward parallel bodies.
	\item[]Under a normal variation $\del_{t}x = f\nu$, these quantities behave just as in the Euclidean space (the effect of the extra term in their definition):
	\eq{\del_{t}W_{k}(\Om) = c_{n,k}\int_{M}H_{k}f\,d\mu.}
	\item[] Wang and Xia proved corresponding quermassintegral inequalities:
	 \end{itemize}
\end{frame}




\begin{frame} \setbeamercovered{invisible}
\frametitle{Almost umbilicity III: Stability in hyperbolic space }
\begin{theorem}[Guofang Wang and Chao Xia\footnotetext{\emph{Isoperimetric type problems and {A}lexandrov-{F}enchel type inequalities in the hyperbolic space, Adv.~Math. {\bf 259} (2014), 532--556.}}]
Let $n\geq 2$, $\Omega \subset \mathbb{H}^{n+1}$ be a horo-convex domain, and $1\leq m\leq n-1$.
\eq{W_{m+1}(\Om)\geq f_{m+1}\circ f_{m}^{-1}(W_{m}(\Om)),}
with equality iff $\Om$ is a geodesic ball. Here $f_{m}(r)=W_{m}(B_{r})$.
\end{theorem}  
Associated with this we get a stability estimate:
\end{frame}


\begin{frame} \setbeamercovered{invisible}
\frametitle{Almost umbilicity III: Stability in hyperbolic space }
\begin{theorem}[with Prachi Sahjwani\footnotetext{\emph{Stability of the quermassintegral inequalities in hyperbolic space, J.~Geom.~Anal. {\bf 34} (2024), art.~13.}}]\label{Stability AF}
Let $n\geq 2$, $\Omega \subset \mathbb{H}^{n+1}$ be a horo-convex domain, and $1\leq m\leq n-1$. Then there exists a constant  $C= C\left(n,\rho_{-}(\Om),\max_{\del\Om}H_{m}/H_{m-1}\right)$ and a geodesic sphere $S_{\bbH}$ such that
\eq{\label{eq:Stability AF}
\mathrm{dist}(\partial \Omega, S_{\mathbb{H}}) \leq C \left(W_{m+1}(\Omega)-f_{m+1}\circ f_{m}^{-1}\left(W_{m}(\Omega)\right)\right)^{\frac{1}{m+2}}.
}
\end{theorem}  

\begin{itemize}
\item[] The dependence on $H_{m}\bs H_{m-1}$ allows for curvature blow-up.
\item[] This is in contrast to the Euclidean case. 
\item[] On the other hand, in the Euclidean case I considered more general hypersurfaces.
\end{itemize}

\end{frame}





















 







\begin{frame}\setbeamercovered{invisible}
\begin{center}\Huge \c{C}ok te\c{s}ekk\"ur ederim! \end{center}

 \end{frame}















\bibliographystyle{amsplain}
\bibliography{/Users/julianscheuer/Documents/Uni/TexTemplates/Bibliography.bib}


\end{document}








